\documentclass[CJKutf8]{beamer}

% This file is a solution template for:

% - Talk at a conference/colloquium.
% - Talk length is about 20min.
% - Style is ornate.


% Copyright 2004 by Till Tantau <tantau@users.sourceforge.net>.
%
% In principle, this file can be redistributed and/or modified under
% the terms of the GNU Public License, version 2.
%
% However, this file is supposed to be a template to be modified
% for your own needs. For this reason, if you use this file as a
% template and not specifically distribute it as part of a another
% package/program, I grant the extra permission to freely copy and
% modify this file as you see fit and even to delete this copyright
% notice. 


\mode<presentation>
{
  \usetheme{Warsaw}
  % or ...

  \setbeamercovered{transparent}
  % or whatever (possibly just delete it)
}

\usepackage{graphicx}

\usepackage{CJKutf8}

\usepackage{CJKulem}

\usepackage{textcomp}

\newenvironment{SC}{%
  \CJKfamily{gbsn}%
  \CJKtilde
  \CJKnospace}{}

\newenvironment{JP}{%
  \CJKfamily{min}%
  \CJKtilde
  \CJKnospace}{}

\usepackage[english]{babel}
% or whatever

%\usepackage[latin1]{inputenc}
% or whatever

\usepackage{times}
%\usepackage[T1]{fontenc}
% Or whatever. Note that the encoding and the font should match. If T1
% does not look nice, try deleting the line with the fontenc.

\begin{CJK}{UTF8}{}
\begin{SC}
% Take \title \author \date from preamble to the document, 
% otherwise CJK will show incorrectly. 
\end{SC}
\end{CJK}

% If you have a file called "university-logo-filename.xxx", where xxx
% is a graphic format that can be processed by latex or pdflatex,
% resp., then you can add a logo as follows:

% \pgfdeclareimage[height=0.5cm]{university-logo}{university-logo-filename}
% \logo{\pgfuseimage{university-logo}}


% Delete this, if you do not want the table of contents to pop up at
% the beginning of each subsection:
\AtBeginSubsection[]
{
  \begin{frame}<beamer>{Outline}
    \tableofcontents[currentsection,currentsubsection]
  \end{frame}
}

% If you wish to uncover everything in a step-wise fashion, uncomment
% the following command: 

%\beamerdefaultoverlayspecification{<+->}


\begin{document}

\begin{CJK}{UTF8}{}
\begin{SC}

\title{2009年7月14日至9月30日工作报告}
\author{Sun~~~Ge}
\date{\today}

\begin{frame}
  \titlepage
\end{frame}

\begin{frame}{Outline}
  \tableofcontents
  % You might wish to add the option [pausesections]
\end{frame}


% Structuring a talk is a difficult task and the following structure
% may not be suitable. Here are some rules that apply for this
% solution: 

% - Exactly two or three sections (other than the summary).
% - At *most* three subsections per section.
% - Talk about 30s to 2min per frame. So there should be between about
%   15 and 30 frames, all told.

% - A conference audience is likely to know very little of what you
%   are going to talk about. So *simplify*!
% - In a 20min talk, getting the main ideas across is hard
%   enough. Leave out details, even if it means being less precise than
%   you think necessary.
% - If you omit details that are vital to the proof/implementation,
%   just say so once. Everybody will be happy with that.

\section{工作小结}

\subsection{岸上实验室工作小结}

\begin{frame}{我在岸上实验室做了哪些工作}%{subtitle is optional}
  % - A title should summarize the slide in an understandable fashion
  %   for anyone how does not follow everything on the slide itself.

  \begin{itemize}
  \item
    累计时间:7月份5工作日、8月份10工作日、9月份5工作日,共计20工作日,约160小时。
    \pause
  \item
    和同学一起提取MC-RR,MC-LR。
  \item
    排除Waters半制备色谱仪故障2次,岛津HPLC故障1次。
  \item
    配制水化试验标准溶液4种1批次。    
  \item
    装填C18富集柱等试验准备工作。
  \item
    阅读菲尼根、Waters、岛津手册及阅读老师专著。
  \end{itemize}

\end{frame}

\begin{frame}{我在岸上试验室学习到了什么?}
通过和同学一起学习,逐步了解几种大型分析仪器的原理,使用、维护、修理方法。

    \begin{itemize}
    \item
      岛津HPLC
      \pause
    \item    
      Waters半制备色谱
      \pause
    \item
      菲尼根LC-MS
    
    \end{itemize}

\end{frame}

%  \item
%    using overlay specifications:
%    \begin{itemize}
%    \item<3->
%      First item.
%    \item<4->
%      Second item.
%    \end{itemize}

\begin{frame}{我在岸上试验室里学习到了什么?}

和同学一起学习,逐步了解了一些设备的原理和使用方法:
    \begin{itemize}
    \item 蒸馏水机、纯水机
    \item 烘箱、马弗炉
    \item 冻干机、真空干燥器
    \item 微型旋转蒸发仪、旋转蒸发仪
    \item 低温摇床、冷冻离心机
    \item 电磁搅拌器、pH计、紫外可见分光计
    \item 超声波细胞破碎仪、离心机、振荡器
    \end{itemize}

\end{frame}

\begin{frame}{我在岸上试验室学习到了什么?}
和同学一起,\emph{初步}了解了一些试验方法和常识
  \begin{itemize}
  \item 微囊藻毒素的提取过程与分析方法
  \item 用酶标仪测定水中叶绿素含量的方法
  \item 试验设备的布置
  \item 试剂、工具、消耗品的位置
  \end{itemize}
\end{frame}

\subsection{太湖站工作小结}

\begin{frame}{我在太湖站做了哪些工作}
  \begin{itemize}
  \item
  累计时间:7月份7工作日,8月份14工作日,9月份19工作日共计40工作日。
  \pause
  \item 全太湖采样2次,4工作日
  \item 贡湖及望虞河望亭、常熟枢纽采样2次
  \item 华庄围隔取样10次
  \item 太湖站水分析试验室勤务34工作日
  \end{itemize}
\end{frame}

\begin{frame}{我在太湖站学到了什么?}
通过向同学学习,初步掌握了湖泊生态学研究的一些方法和常识
  \begin{itemize}
  \item 使用有机玻璃采水器采集水样
  \item 使用浮游生物网采集浮游生物
  \item 测定水深
  \item 测定透明度
  \item 测定水温
  \item 需要现场进行的前处理项目
    \begin{itemize}
       \item 使用浮游生物网过滤湖水
       \item 使用福尔马林固定水中的微生物
       \item 使用硫酸锰和碱性碘化钾固定水中的溶解氧
    \end{itemize}
  \item 天气状况记录、现场摄影以及其他用传感器测定的项目
  \item 使用采泥器采集底泥
  \end{itemize}
\end{frame}

\begin{frame}{我在太湖站学到了什么?}
和同学一起学习,初步了解了太湖水环境监测的常识
  \begin{itemize}
  \item 太湖地理、气候、周边环境概况
  \item 监测点与监测频度
  \item 测定指标
  \item 对7、8月蓝藻水华状况有了一定了解
  \end{itemize}
\end{frame}

\begin{frame}{我在太湖站学到了什么?}
向同学请教,逐步掌握或强化了一些试验方法
  \begin{itemize}
  \item 水样的前处理方法
     \begin{itemize}
     \item 异味物质试样
     \item 叶绿素试样
     \item 微囊藻毒素试样
     \item 显微镜检试样
     \end{itemize}
  \item 水化学指标的测定方法
  \item pH、COD测定法
  \item 722型光栅分光光度计使用法
  \item 三洋高压蒸汽灭菌锅使用法
  \item 岛津紫外/可见分光光度计使用法
  \end{itemize}
\end{frame}

\begin{frame}{我在太湖站做了哪些工作方法的改进?}
  \begin{block}{总氮测定:试样消解时比色管的封口方法改进}
    在马师姐指导下改进的方法是,采用铝箔(锡箔纸)覆盖比色管口,边缘用手压紧,不留空隙,在铝箔上标记。原封口方法为,管口用玻璃塞塞紧,外包有记号的纱布,用橡皮筋固定。
  \end{block}
  \begin{itemize}
  \item 经试验,两种封口方法在消解时损失的水分是一样多的
  \item 推断两种方法消解时损失的氨态氮也相同
  \item 铝箔封口能节省2/3的封口时间
  \item 铝箔封口能节省去掉纱布、第二次标记号码的时间
  \end{itemize}
\end{frame}

\section{对岸上试验室的改进建议}

\subsection{现状}

\begin{frame}
  \begin{exampleblock}{岸上试验室的不足之处(硬件)}
    \begin{itemize}
    \item 试剂摆放混乱,柜子外没有内容标记
    \item 一种试剂有还是一瓶都没有了,很难确定
    \item 玻璃器皿、用具、消耗品摆放比较混乱
    \item 放在抽屉里的是什么东西,不打开抽屉就不知道
    \item 试验台上的器具沾染了灰尘,可是却有一些空抽屉
    \item 需要不时取用的东西要花很长时间找到
    \end{itemize}
  \end{exampleblock}
\end{frame}

\begin{frame}
  \begin{exampleblock}{岸上试验室的不足之处(软件)}
    \begin{itemize}
    \item 做水化试验找不到SOP
    \item 用分析天平、pH计,找不到校正方法
    \item 用不熟悉的仪器时找不到使用说明
    \end{itemize}
  \end{exampleblock}
\end{frame}

%\begin{frame}{为什么会这样?全世界人民震惊了}
%\includegraphics{shock.jpg}
%\end{frame}

\subsection{改进建议}

\begin{frame}{硬件:没有试剂柜怎么办?}
  \begin{block}{解决方案}
    \begin{enumerate}
    \item 将试剂分类
    \item 不常用试剂、固体试剂、避光试剂放入试验台下的柜子
    \item 在柜子外面贴上标签,依照摆放次序标记内容物
    \item 常用液体试剂集中摆放在靠近通风橱的试验台上
    \end{enumerate}
  \end{block}
  \pause
  这样做的好处有:
  \begin{itemize}
  \item 不需要打开柜子就知道内容物
  \pause
  \item 减少下蹲次数
  \pause
  \item 节省找试剂的时间
  \pause
  \item 如果试剂用完了能及时发现和订购
  \end{itemize}
\end{frame}

\begin{frame}{硬件:要用时立刻得到干净的器皿!(通用篇)}
  \begin{block}{解决方案}
    \begin{enumerate}
    \item 把试验台的抽屉清洁干净
    \item 把烧杯、圆底烧瓶、三角瓶、量筒、容量瓶、移液管、胶头滴管、称量瓶等玻璃器皿清洗干净、烘干或晾干
    \item 把干燥、干净的玻璃器皿分别存放于抽屉内
    \item 在抽屉外做好内容物标记
    \item 把干燥、干净的称量瓶放入干燥器中
    \item 使用器皿后,清洗,放入烘箱内或烘箱上的篮子里。
    \pause
    \item 每天检查一次烘箱和烘箱上的篮子,把干燥好的玻璃器皿放入我们设定的地点
    \end{enumerate}
  \end{block}
\end{frame}

\begin{frame}{硬件:要用时立刻得到干净的器皿!(通用篇)}
  这样做的好处有:
  \begin{itemize}
  \item 立即从抽屉中得到需要的器皿
  \pause
  \item 充分信任器皿的洁净程度
  \pause
  \item 如果器皿不足能即时发现和订购
  \end{itemize}
\end{frame}

\begin{frame}{硬件:要用时立刻得到干净的器皿!(专用篇)}
  \begin{block}{解决方案}
    \begin{enumerate}
    \item 把微孔过滤用瓶、旋转蒸发用连接器、流动相配制用量筒等为单一过程服务、且不会交叉污染的器皿作为专用器皿
    \item 将专用器皿放入抽屉或者柜子中存放
    \item 在抽屉或者柜子外做标记
    \item 专用器皿使用前后一般不需要清洗
    \item 对灰尘敏感的器皿,可以用铝箔封口
    \end{enumerate}
  \end{block}
  \pause
  这样做的好处有:
  \begin{itemize}
  \item 节省时间和精力
  \pause
  \item 避开灰尘
  \end{itemize}
\end{frame}

\begin{frame}{硬件:就近取用小工具}
  \begin{block}{解决方案}
    \begin{enumerate}
    \item 将一次性移液管头放在移液器下面的抽屉中
    \item 将搅拌子放在小微波盒里,放在电磁搅拌器下面的抽屉中
    \item 将裁减好的滤纸放在小微波盒里,放在色谱仪下的抽屉中
    \item 将称量纸放在小微波盒里,放在分析天平下的抽屉中
    \item 将药匙放在小微波盒里,放在分析天平下的抽屉中
    \item 将扳手等工具放在塑料盒子里,放在色谱仪下面的抽屉中
    \item 将防热手套放在马弗炉、烘箱下的抽屉中
    \item 将剪刀、封口膜、标签纸、电工胶带、文具集中放在篮子里,放在试验台上。
    \end{enumerate}
  \end{block}
\end{frame}
%
\begin{frame}{硬件:就近取用小工具}
  \begin{block}{每天观察烘箱一次}
  对于非一次性的、且需要保持洁净的小工具,如药匙、电磁搅拌子等,应采取和玻璃器皿一样的清洗、还原方法。即用完洗净后放入烘箱中。每天应至少打开烘箱一次,把干燥的器皿和小工具放入原抽屉中。
  \end{block}
  这样做的好处有:
  \begin{itemize}
  \item 不用离开实验台面就能取用工具
  \pause
  \item 随时用到干净的工具
  \pause
  \item 文具、封口膜、胶带、标签纸等小物易找到不易丢失
  \end{itemize}
\end{frame}

\begin{frame}{软件:立即读到实验设备的使用说明}
  \begin{block}{解决方案}
    \begin{enumerate}
    \item[1] 整理实验设备
    如超低温冰箱、蒸馏水机、马弗炉、
    紫外分光计、分析天平、真空干燥箱、
    冻干机、Apex微波反应器、微型旋转蒸发仪、
    旋转蒸发仪、烘箱、超声波细胞破碎仪、
    pH计、离心机、冷冻离心机、低温摇床的使用说明。
    \end{enumerate}
  \end{block}
\end{frame}

\begin{frame}{软件:立即读到实验设备的使用说明}
 \begin{block}{解决方案}
    \begin{enumerate}
    \item[2] 提取基本的操作步骤,\textbf{打印到一页纸上}。纸张最大不超过A4。
    \item[3] 分析天平、pH计应包括校正方法。或单独打印校正步骤。
    \item[4] 将印有操作步骤的纸张过塑。
    \item[5] 如果设备的正面或侧面有位置,则把操作步骤用透明胶带贴在设备上。
    \item[6] 如果纸张太大,则把操作说明夹在硬质塑料文件夹板上,放在设备附近的台面上。
    \end{enumerate}
  \end{block}
  \pause
  这样做的好处有:
  \begin{itemize}
  \item 节省时间,用时立即获得准确的操作方法
  \pause
  \item 节省精力,不必要记住某些操作或校正方法。
  \pause
  \item 最大程度避免操作错误。
  \end{itemize}
\end{frame}

\begin{frame}{软件:立即读到水化学实验的操作指南}
  \begin{block}{解决方案}
    \begin{enumerate}
    \item 根据书上的标准操作规程和实际情况,编辑一份易懂的操作指南
    \item 如果没有编辑的时间,则直接复印书上的标准操作规程,如有必要,可以用笔修改
    \item 打印或复印若干份,每一份放进一个文件夹里。
    \item 文件夹外写上标题,目录。
    \item 把文件夹放在LC-MS室的架子上。
    \item 出差时可带走一份。
    \end{enumerate}
  \end{block}
\end{frame}

\begin{frame}{软件:立即读到水化学实验的操作指南}
  这样做的好处有:
  \begin{itemize}
  \item 节省记住方法的细节的精力
  \pause
  \item 最大程度避免操作错误。
  \pause
  \item 有标准操作规程、又不要求记录过程的实验,则可以只在实验笔记本上记录异常、结果和结果分析三部分,节省大量时间。
  \end{itemize}
\end{frame}

\begin{frame}{制度:在岸上试验室导入星期五清洁制}
  \begin{block}{星期五清洁制度}
  星期五清洁制度是若干人一起,每周做一次清洁(比如周五)。每周轮换某个人的任务。
  \end{block}
  \begin{exampleblock}{采用星期五清洁制度的例子}
    \begin{itemize}
    \item 比如某个星期五安排甲乙丙丁戊做清洁。

    \item 甲换超声波仪的水

    \item 乙换旋转蒸发仪的水

    \item 丙处理有机废液

    \item 丁整理试验台面

    \item 戊清洗水槽

    \item 到下一周的周五,每个人的工作会轮换
    \end{itemize}
  \end{exampleblock}
\end{frame}

\begin{frame}{制度:在岸上试验室导入星期五清洁制}
这样做的好处是:
  \begin{itemize}
  \item 大部分清洁工作不需要每天去做
  \pause
  \item 每周找一段时间大家一起做,能有效地调动积极性
  \pause
  \item 节省时间
  \end{itemize}
\end{frame}

\begin{frame}{制度:导入分析天平、pH计校正制度}
  \begin{block}{导入分析天平校正制度}
    \begin{enumerate}
    \item 找出砝码
    \item 打印分析天平校正记录表
    \item 把校正记录表夹到天平校正步骤的硬塑料文件夹板上
    \item 每张校正记录表可以填写1个月
    \item 每天第一个使用分析天平的人,要用砝码反测分析天平,看是否在容许值内
    \item 如果需要校正,则按校正步骤进行校正
    \item 填写校正记录,如果不用校正,填写砝码的反测值
    \item 如果校正过,要填写校正后砝码的反测值。
    \item 用完天平后不需要关机,每日试验完毕后关机
    \item 每天校正步骤需花费2-5分钟。
    \end{enumerate}
  \end{block}
\end{frame}

\begin{frame}{制度:导入分析天平、pH计校正制度}
  这样做的好处有:
  \begin{itemize}
  \item 以1天1人2-5分钟的极低成本换取数据的准确
  \pause
  \item 及时发现分析天平的故障
  \end{itemize}
\end{frame}

\begin{frame}{制度:导入分析天平、pH计校正制度}
  \begin{block}{导入pH计校正制度}
    \begin{enumerate}
    \item 找出标准缓冲溶液
    \item 打出pH计校正记录表
    \item 把校正记录表夹到pH计校正步骤的硬塑料文件夹板上
    \item 每张校正记录表可以填写1-6个月
    \item 每个使用pH计的人,要先用标准缓冲溶液校正pH计
    \item 校正完毕后,填写校正记录
    \item 每次校正步骤需要5分钟
    \end{enumerate}
  \end{block}
  \pause
  这样做的好处有:
  \begin{itemize}
  \item 以1次5分钟的极低成本换取数据的准确
  \pause
  \item 及时发现pH计的故障
  \end{itemize}
\end{frame}

\section{总结}

\begin{frame}{中心思想:不要在不需要创造力的地方发挥创造力!}
对岸上试验室进行便利化和规范化改进,其目的是:
  \begin{itemize}
  \item
    \alert{让试验过程消耗大家最小的时间、体力和脑力}。
  \pause
  \item
    \alert{让思维定势来主导试验中不需要创造力的部分}。
  \pause
  \item
    \alert{让大家把精力集中到创造性环节中}。
  \pause
  \item
    \alert{不让可疑的数据来分散脑力和消耗时间}。
  \end{itemize}
  
\pause
  % The following outlook is optional.
  \vskip0pt plus.5fill
  \begin{itemize}
  \item
    下阶段工作展望
    \begin{itemize}
    \item
      继续向大家学习各方面的理论和实践知识。
    \item
      逐步推进岸上试验室的改进工作。
    \end{itemize}
  \end{itemize}
\end{frame}


% All of the following is optional and typically not needed. 

\end{SC}
\end{CJK}
\end{document}
