%JP15微生物限度试验法中文翻译稿.
%todo: 增加〔注〕项。
%采用latex->dvips->ps2pdf产生的pdf文档字体较pdflatex产生字体为佳.

\documentclass[11pt,a4paper]{article}

%\documentclass[10pt,a4paper,twocolumn]{article}
%\usepackage{fullpage}
%如果使用两栏,虽然紧凑,但表格总是太宽,放不下.

\usepackage{indentfirst}
\usepackage{endnotes}
\usepackage{fancybox}

\usepackage{CJKutf8}

%使用\uline
\usepackage{CJKulem}

\usepackage{textcomp}

\newenvironment{SC}{%
  \CJKfamily{gbsn}%
  \CJKtilde
  \CJKnospace}{}

\newenvironment{JP}{%
  \CJKfamily{min}%
  \CJKtilde
  \CJKnospace}{}

%\pagestyle定义在CJK外。
\pagestyle{myheadings}

\begin{CJK}{UTF8}{}
\begin{SC}
%定义heading,因为是article,所以与book不同。
\markboth{}{\uline{微生物限度试验法}}

\title{4.05~~微生物限度试验法}
%\author{译者:sun.ge@126.com}
\date{}
\end{SC}
\end{CJK}

\begin{document}

\begin{CJK}{UTF8}{}
\begin{SC}

% familiar with original:
%\renewcommand{\makeenmark}{\hbox{$^\theenmark$}}
% common box with frame:
%\renewcommand{\makeenmark}{\framebox[1.1\width]{\footnotesize{
%\textbf{注\theenmark}}}}

%重新定义notesname.
\renewcommand{\notesname}{\begin{center}------注------\end{center}}
% oval box (use fancybox package):
\renewcommand{\makeenmark}{\ovalbox{\footnotesize{\textbf{注\theenmark}}}}

\maketitle
%使第一页也有heading。
\thispagestyle{myheadings}
微生物限度试验法是在医药品等中存在的有繁殖能力的特定的微生物的
定性和定量试验法。本试验法包括活菌数试验(细菌及真菌)及特
定微生物试验(大肠杆菌、沙门氏菌、绿脓杆菌及金
黄色葡萄球菌)。试验必须在避免引起外部的微生物污染的条件下施行,
必须仔细操作。并且,当被检试料有抗菌作用或含有抗菌物质,任何这样
的抗菌性质必须被稀释、过滤、中和或失活等手段除去。用从该种原料或制品
的随机选择的不同地点(或部分)的采样物混合,作为试料进行试验。
若试料用液体培养基稀释,试验应
快速进行。并且,在试验过程中,必须十分注意防止生物危害。\endnote{
\begin{JP}ハイオハザード\end{JP}(biohazard)译作生物学的危害,
这里指人或周边的生物受到病原微生物的危害。}
%用\begin{JP}...\end{JP}插入日本语。

\section{活菌数试验}
本试验测定在需氧条件下生长的中温性细菌及真菌(霉菌及酵母)。
低温菌、高温菌、好盐菌\endnote{}、厌气菌\endnote{}及需要特定生长成分的
微生物,即使在试料中大量存在,也可能得出阴性结果\endnote{}。
本试验法有膜过滤法、稀释混匀平板法、涂
布平板法及液体培养基逐级稀释法(最确数法)四种。
依照目的,采用适当的方法进行试验。如果自动化的方法与这里提供的
方法相比,能够得出同等或更好的检出感度与精度,则也可以被采用\endnote{}。
细菌和真菌(霉菌与酵母)需要不同的培养基及培养温度。液体培养基逐级稀释法
(最确数法)仅在细菌中应用。

\subsection*{试料溶液的制备}
使用pH7.2的磷酸缓冲液、pH7.0的氯化钠蛋白胨缓冲液,或使用液体培养基
来分散或稀释试料。除非另有规定,使用10g或10mL试料。
但是,由于试料的性质,也有不得不使用与其相异的量的情况。调节试料溶液的pH
值为6$\sim$8\endnote{}。试料溶液必须在制备后1小时内使用。\\
\indent液装制剂与可溶性固形剂:取10mL试料,加入上述缓冲液或液体培养基中至100mL,
振摇混合作为试料溶液。含有不溶性物质的液体制剂的情况下,混合前应时时充分
振摇使均一化。\\
\indent不溶性固形剂:取10g试料,将不溶性物质磨碎成合适细粉,加入上述
缓冲液或液体培养基中至100mL,振摇混合作为试料溶液。但是,也有由于试料的性质,
不得不分散于比规定量大的缓冲液或液体培养基中的
情况。如有必要,使用机械搅拌器可能会使悬浊液均一分散得更好。可以加入
适当的表面活性剂(比如$0.1w/v\%$的吐温80)辅助溶解。\\
\indent脂质制品:以脂质为主要构成物质的软膏、乳膏、蜡剂、洗剂等的半固形剂
及液体制剂等,取10g或10mL,用吐温20或吐温80这样的表面活性剂辅助,加入上述
缓冲液或液体培养基中乳化至100mL,即为试料溶液。这种情况可以加热在45\textcelsius
以下的温度乳化。但是,不要将试料加热超过30分钟。\endnote{}

%重新定义enmuerate的label为(1),(2)...
%\renewcommand{\labelenumi}{(\arabic{enumi})}

\subsection*{试验过程}
%重新定义subsubsection的label为(1),(2)...
\renewcommand{\thesubsubsection}{(\arabic{subsubsection})}
\subsubsection{膜过滤法} 
本法特别适用于在试料中含有抗菌物质的情况下将其除去。
使用孔径0.45$\mu$m以下的适当材质的滤膜\endnote{}。滤膜直径建议为约50mm,但不同直径的
也可以使用。滤膜、过滤装置、培养基必须全部仔细灭菌。通常,取20mL试料溶液
(含2g试料),用2枚滤膜过滤,每枚10mL。必要时稀释预先制备的试料溶液。若
微生物浓度高则稀释至1枚滤膜对应的菌数为10$\sim$100CFU的期望水平。试料溶液滤过后,
各滤膜用pH7.0的氯化钠蛋白胨缓冲液、pH7.2的磷酸缓冲液或要使用的液体培养基等
洗液清洗过滤3次以上。每次清洗过滤洗液量约为100mL,但若滤膜直径不为50mm,则
按照尺寸调整洗液的量。若试料含有脂类物质,则洗液中可添加吐温80等。滤过后,
将1枚滤膜放在SCD琼脂培养基平板\endnote{}表面上,用作细菌试验,
将1枚滤膜放在Sabouraud葡萄糖琼脂培养基、马铃薯葡萄糖琼脂培养基或GP琼脂培养基
(全部添加抗生素)\endnote{}平板表面上,用作真菌试验。细菌试验于30$\sim$35\textcelsius,真菌
试验于20$\sim$25\textcelsius,不少于5天培养后,计算菌落数。如果在培养后5天以前能够获取一个
稳定的计数值,那么也可以采用该计数值。

\subsubsection{稀释混匀平板法}
本法使用直径9$\sim$10cm的培养皿。每一稀释级使用2枚以上的琼脂培养基平板。
各吸取1mL的试料溶液或试料溶液的稀释液分注于无菌培养皿上。往皿中加入
预先制备好的45\textcelsius以下保温的融化状态己灭菌琼脂培养基15$\sim$20mL,混合。
使用SCD琼脂培养基检出细菌,
使用添加抗生素的Sabouraud葡萄糖琼脂培养基、添加抗生素的
马铃薯葡萄糖琼脂培养基或者添加抗生素的GP琼脂培养基检出真菌。
琼脂固化后,细菌的平板于30$\sim$35\textcelsius,真菌的平板于20$\sim$25\textcelsius培养至少
5天。在出现许多菌落的情况下,细菌采用每皿菌落数300CFU以下的
平板,真菌采用每皿菌落数100CFU以下的平板上的计数结果算出活菌数。
如果在培养后5天以前的时间里能够获取一个稳定的计数值,也可以采用该
计数值。

\subsubsection{涂布平板法}
本法为在固化干燥的琼脂培养基表面,注入0.05$\sim$0.2mL的试料溶液,用涂布
棒等均匀涂布于表面的方法。培养皿的直径、使用琼脂培养基的种类与用量、
培养的温度和时间及活菌的计数方法等,与稀释混匀平板法相同。

\subsubsection{液体培养基逐级稀释法(最确数法)\endnote{}}
本法需要准备12支每支装有9$\sim$10mL液体SCD培养基的试管。各稀释级使用3支试管。
对第1级的3支试管,每支各加入试料溶液1mL(含0.1g或0.1mL试料),即为
10倍稀释的试管。下一步从10倍稀释的试管中各取1mL,分别加入3支试管
中混合,即为100倍稀释的试管。再从100倍稀释的试管中各取1mL,分别加入
3支试管中混合,即为1000倍稀释的试管。
取各稀释级的稀释液1mL同上法加入剩余的3支试管中,即为对照。上述试管于
30$\sim$35\textcelsius
培养不少于5天,对照试管应观察无微生物生长。若结果判定困难或结果含混,
则取0.1mL接种至琼脂培养基或液体培养基中,
于30$\sim$35\textcelsius培养24$\sim$72小时,判定
有无生长。依照表4.05-1求出1g或1mL试料中相应的最确数。若第1列(含0.1g或
0.1mL试料)观察到微生物生长的试管数为2及以下,则1g或1mL对应的微生物的
最确数很可能在100以下。

\begin{table}[tb]
\begin{center}
表4.05-1 微生物的最确数表\\
\begin{tabular}{c|c|c|c}
\hline
\multicolumn{3}{c|}{加入以下数量的试料的情况下,}&试料1g或1mL中相当\\
\multicolumn{3}{c|}{观察到微生物生长的试管数}&的微生物的最可能数\\
\cline{1-3}
相当于0.1g&相当于0.01g&相当于1mg\\
或0.1mL的试管&或0.01mL的试管&或1uL的试管\\
\hline
3&3&3&$>$1100\\
3&3&2&1100\\
3&3&1&500\\
3&3&0&200\\
\hline
3&2&3&290\\
3&2&2&210\\
3&2&1&150\\
3&2&0&90\\
\hline
3&1&3&160\\
3&1&2&120\\
3&1&1&70\\
3&1&0&40\\
\hline
3&0&3&95\\
3&0&2&60\\
3&0&1&40\\
3&0&0&23\\
\hline
\end{tabular}
\end{center}
\end{table}

\subsection*{培养基性能试验及抗菌物质的确认试验}
使用以下菌株或其同等品。使用SCD培养基,细菌于30$\sim$35\textcelsius,
\textsl{Candida albicans}于20$\sim$25\textcelsius培养。\endnote{}\\
%\begin{table}[h]
%\begin{center}
\begin{tabular}{ll}
\textsl{Escherichia coli}	&NBRC 3972,ATCC 8739,NCIMB 8545等\\
\textsl{Bacillus subtilis}	&NBRC 3134,ATCC 6633,NCIMB 8054等\\
\textsl{Staphylococcus aureus}	&NBRC 13276,ATCC 6538,NCIMB 9518等\\
\textsl{Candida albicans}	&NBRC 1393,NBRC 1594,ATCC 2091,ATCC 10231等\\
\end{tabular}
%\end{center}
%\end{table}
用pH7.0的氯化钠蛋白胨缓冲液或pH7.2的磷酸缓冲液分别稀释各培养液,制备1mL对应
有50$\sim$200CFU左右活菌的的菌悬液。将1mL菌悬液接种至要使用的培养基中,于指定
温度培养5天,应观察到充分生长或确认回收到接种菌数。当试料存在下的菌数比无
试料存在的菌数少1/5以下时,任何这样的效应必须被稀释、过滤、中和或失活等
手段除去。为了验证培养基、稀释液的无菌性或试验是否为无菌操作,用pH7.0的
氯化钠蛋白胨缓冲液或pH7.2的磷酸缓冲液作为试验对照。\\
%
\section{特定微生物试验}
本试验为测定大肠杆菌、沙门氏菌、绿脓杆菌及金黄色葡萄球菌
的试验。在本试验中作为检出目的的4种微生物,不仅在最终制品中,而且在原料及
制造过程的中间体等中,对于微生物污染的评估都是非常重要的,并且,是不应在
上述物料中存在的微生物的代表。\endnote{}

\subsection*{试料溶液的制备}
除非另有规定,参照活菌数试验中的试料溶液的制备项。使用液体培养基来溶解或
稀释试料的情况下,除非另有规定,使用在该试验中指定的培养基。

\subsection*{试验过程}
\setcounter{subsubsection}{0}
\subsubsection{大肠杆菌\endnote{}}
取试料10g或10mL,加入乳糖肉汤至100mL,
于30$\sim$35\textcelsius培养24$\sim$72小时。
若观察到生长,轻轻振摇培养液后,取接种环等,划线于麦康凯琼脂培养基表面,
于30$\sim$35\textcelsius培养18$\sim$24小时。
若没有发现环绕有带红色的沉淀区的砖红色革兰氏阴性菌的菌落,
则判定大肠杆菌阴性。若检出持有上述特征的菌落,则取其菌落,划线于
EMB琼脂培养基表面,于30$\sim$35\textcelsius培养18$\sim$24小时。
若EMB琼脂培养基上未发现有金属
光泽的且透过光下呈蓝黑色的菌落,则判定为大肠杆菌阴性。上述平板上有疑似
大肠杆菌的菌落时,进行IMViC试验\endnote{}
(吲哚试验、甲基红试验、V-P试验及柠檬酸盐利用试验),
结果出现「$+$ $+$ $-$ $-$」或「$-$ $+$ $-$ $-$」模式时判定检出大肠杆菌。另外,
也可以使用大肠杆菌快速测定培养基与工具包来确定。
\subsubsection{沙门氏菌\endnote{}}
取试料10g或10mL,加入乳糖肉汤至100mL,于30$\sim$35\textcelsius
培养24$\sim$72小时。若观察到生长,轻轻振摇培养液,然后分别吸取1mL接种至10mL的
液体亚硒酸盐胱氨酸培养基、液体四硫磺酸盐培养基中,培养12$\sim$24小时。另外,
液体亚硒酸盐胱氨酸培养基可用液体Rappaport培养基代替使用。培养后,分别从上述
液体培养基中取培养液划线于亮绿琼脂培养基、XLD琼脂培养基及亚硫酸铋琼脂
培养基中至少2种培养基表面,于30$\sim$35\textcelsius培养24$\sim$48小时。
若没有出现符合表4.05-2所描述的菌落,判定沙门氏菌阴性。
当出现表4.05-2所描述的
革兰氏阴性杆菌的菌落时,用接种线取可疑菌落于TSI斜面琼脂培养基深部及表面
接种,于35$\sim$37\textcelsius培养18$\sim$24小时。若存在沙门氏菌,则培养基深部变黄,斜面则仍
为红色无变化\endnote{}。通常深部产气,产生或不产生硫化氢。若使用鉴定工具包中包括的
详细的生化试验和血清学试验,能进行对沙门氏菌的识别和分类。
\begin{table}[htb]
\begin{center}
表4.05-2 选择培养基上的沙门氏菌形态学特征\\
\begin{tabular}{c|l}
\hline
培养基&菌落特征\\
\hline
亮绿琼脂培养基&小型、无色透明或不透明、白色至粉红色\\
&常常环绕有一个粉红至红色的区域\\
\hline
XLD琼脂培养基&红色,中心有黑点或者没有\\
\hline
亚硫酸铋琼脂培养基&黑色或绿色\\
\hline
\end{tabular}
\end{center}
\end{table}
\subsubsection{绿脓杆菌\endnote{}}
取试料10g或10mL,加入SCD培养基或不含抗菌物质的适宜培养基至100mL。
乳糖肉汤不适合使用。将含有试料的液体培养基于30$\sim$35\textcelsius
培养24$\sim$48小时。若观察到生长,用接种环等,划线于十六烷三甲基溴化铵琼脂
培养基或NAC琼脂培养基上,于30$\sim$35\textcelsius培养24$\sim$48小时。
若没有观察到微生物生长,则判定绿脓杆菌阴性。若出现产生带绿色的荧光物质的
革兰氏阴性杆菌的菌落,则划线于荧光素检出用假单胞菌属琼脂培养基及绿脓菌素检出
用假单胞菌属琼脂培养基表面,于30$\sim$35\textcelsius培养24$\sim$72小时。
若在前者上产生黄色荧光物质,则荧光素阳性,若在后者上产生蓝色荧光物质,
则判定绿脓菌素阳性。对怀疑为绿脓杆菌的菌落进行氧化酶试验。
取可疑菌落,移入饱和有二氯化--N,N--二甲基--对苯二胺的滤纸上。若5$\sim$10秒
以内变为紫色,则判定氧化酶试验阳性。若氧化酶试验阴性,则判定绿脓杆菌阴性。
也可以使用含有适当的生化试验的工具包来确认绿脓杆菌的存在。
\subsubsection{金黄色葡萄球菌\endnote{}}
取试料10g或10mL,加入SCD培养基或不含抗菌物质的适宜培养基至100mL。
将含有试料的液体培养基于30$\sim$35\textcelsius培养24$\sim$48小时。
若观察到生长,
则用接种环等,划线于Vogel-Johnson琼脂培养基、Baird-Parker琼脂培养基
或甘露醇氯化钠琼脂培养基任一培养基表面,于30$\sim$35\textcelsius培养
24$\sim$48小时。若没有革兰氏阳性球
菌的菌落与表4.05-3所描述的特征相附,则判定金黄色葡萄球菌阴性。若有疑似
金黄色葡萄球菌的菌落则进行血浆凝固酶试验\endnote{}。用接种环取可疑菌落接种于装有0.5mL
哺乳类的血浆(最好来源于兔或马;可以加入适当的添加物)的试管中,于37$\pm$1\textcelsius
恒温槽中培养,3小时后检查有无凝固,之后,至24小时每隔适当的时间观察有无
凝固。同时进行血浆凝固反应的阳性及阴性对照试验。若没有观察到凝固,则判定
金黄色葡萄球菌阴性。\\

\begin{table}[htb]
\begin{center}
表4.05-3 选择培养基上的金黄色葡萄球菌形态学特征\\
\begin{tabular}{l|l}
\hline
培养基&菌落特征\\
\hline
Vogel-Johnson琼脂培养基&黑色,环绕有一个黄色的区域\\
\hline
Baird-Parker琼脂培养基&黑色有光泽,环绕有一个透明的区域\\
\hline
甘露醇氯化钠琼脂培养基&黄色菌落,环绕有一个黄色的区域\\
\hline
\end{tabular}
\end{center}
\end{table}

\subsection*{培养基的性能试验及抗菌物质的确认试验}
本试验将表4.05-4所述菌株于规定培养基中,于30$\sim$35\textcelsius培养18$\sim$24小时使用。
下一步,用pH7.0的氯化钠蛋白胨缓冲液、pH7.2的磷酸缓冲液或该菌株指定的培养基
等,配制1mL含相当于约1000CFU的活菌的溶液。当需要时,取含约1000CFU/mL的活菌
的大肠杆菌、沙门氏菌、绿脓杆菌及金黄色葡萄球菌的悬液各0.1mL混合,进行在试料存在和不
存在下的培养基的有效性及是否存在抗菌物质等试验。\endnote{}\\
\begin{table}[htb]
\begin{center}
表4.05-4 培养基的有效性确认与特定微生物试验法的验证所使用的菌株及培养基\\
\begin{tabular}{l|p{3 in}|l}
\hline
微生物&菌株名&培养基\\
\hline
大肠杆菌&NBRC 3972,ATCC 8739,NCIMB 8545\\
&或与其同等的菌株&乳糖肉汤\\
\hline
沙门氏菌&无特定$^{\ast}$&乳糖肉汤\\
\hline
绿脓杆菌&ATCC 9027,NCIMB 8626,NBRC 13275\\
&或与其同等的菌株&SCD培养基\\
\hline
金黄色葡萄球菌&NBRC 13276,ATCC 6538,NCIMB 9518\\
&或与其同等的菌株&SCD培养基\\
\hline
\end{tabular}
\end{center}
$\ast$沙门氏菌以使用无病原性或弱病原性的菌株为宜。\\
最好不要使用\textsl{Salmonella typhi}。\endnote{}\\
\end{table}

\subsection*{再试验}
若得出不确定的结果或含混的结果,使用25g或25mL试料再次试验。试验方法
与最初的试验方法相同,但按照试料增加的比例,增加培养基等的量。\\

\setcounter{subsubsection}{0}
\renewcommand{\labelenumi}{(\roman{enumi})}
\section{缓冲液与培养基\endnote{}}
以下所述为微生物限度试验用的缓冲液、培养基和试药。含有相似营养成分,且
对于试验对象微生物有类似的选择性及增殖性的其它培养基也可以使用。\\
\subsubsection{缓冲液}
\begin{enumerate}
\item 磷酸缓冲液,pH7.2\\
保存溶液:取磷酸二氢钾34g溶于约500mL水中。加入氢氧化钠试液约175mL,调节
pH7.1$\sim$7.3,加水至1000mL,即为保存溶液。高压蒸气灭菌后,于冷处保存。用时,
将保存溶液稀释800倍,于121\textcelsius灭菌15$\sim$20分钟。
\item 氯化钠蛋白胨缓冲液,pH7.0\\
\begin{tabular*}{3in}{l@{\extracolsep{\fill}}r}
磷酸二氢钾&3.56g\\
磷酸氢二钠十二水合物&18.23g\\
氯化钠&4.30g\\
胨&1.0g\\
水&1000mL\\
\end{tabular*}
\\
混合所有成分,于121\textcelsius高压蒸气灭菌15$\sim$20分钟。灭菌后的pH6.9$\sim$7.1,可添加
$0.1\sim1.0w/v\%$的吐温20或吐温80。
\end{enumerate}

\subsubsection{培养基}
\begin{enumerate}
\item SCD琼脂培养基\\
\begin{tabular*}{3in}{l@{\extracolsep{\fill}}r}
酪蛋白胨&15.0g\\
大豆胨&5.0g\\
氯化钠&5.0g\\
琼脂&15.0g\\
水&1000mL\\
\end{tabular*}
\\
混合所有成分,于121\textcelsius高压蒸气灭菌15$\sim$20分钟。灭菌后的pH7.1$\sim$7.3。
\item SCD培养基\\
\begin{tabular*}{3in}{l@{\extracolsep{\fill}}r}
酪蛋白胨&17.0g\\
大豆胨&3.0g\\
氯化钠&5.0g\\
磷酸氢二钾&2.5g\\
葡萄糖&2.5g\\
水&1000mL\\
\end{tabular*}
\\
混合所有成分,于121\textcelsius高压蒸气灭菌15$\sim$20分钟。灭菌后的pH7.1$\sim$7.5。
\item 添加抗生素的Sabouraud葡萄糖琼脂培养基\\
\begin{tabular*}{3in}{l@{\extracolsep{\fill}}r}
胨(动物组织制及酪蛋白制)&10.0g\\
葡萄糖&40.0g\\
琼脂&15.0g\\
水&1000mL\\
\end{tabular*}
\\
混合所有成分,于121\textcelsius高压蒸气灭菌15$\sim$20分钟。灭菌后的pH5.4$\sim$5.8。
使用前每1L培养基直接加入青霉素钾0.10g与四环素0.10g作为灭菌溶液。
作为青霉素钾与四环素的替代品,也可以每1L培养基中加入50mg氯霉素。
\item 添加抗生素的马铃薯葡萄糖琼脂培养基\\
\begin{tabular*}{3in}{l@{\extracolsep{\fill}}r}
马铃薯浸出物&4.0g\\
葡萄糖&20.0g\\
琼脂&15.0g\\
水&1000mL\\
\end{tabular*}
\\
混合所有成分,于121\textcelsius高压蒸气灭菌15$\sim$20分钟。灭菌后的pH5.4$\sim$5.8。
使用前每1L培养基直接加入青霉素钾0.10g与四环素0.10g作为灭菌溶液。
作为青霉素钾与四环素的替代品,也可以每1L培养基中加入50mg氯霉素。
\item 添加抗生素的GP(葡萄糖蛋白胨)琼脂培养基\\
\begin{tabular*}{3in}{l@{\extracolsep{\fill}}r}
葡萄糖&20.0g\\
酵母浸出物&2.0g\\
硫酸镁七水合物&0.5g\\
磷酸二氢钾&1.0g\\
琼脂&15.0g\\
水&1000mL\\
\end{tabular*}
\\
混合所有成分,于121\textcelsius高压蒸气灭菌15$\sim$20分钟。灭菌后的pH5.4$\sim$5.8。
使用前每1L培养基直接加入青霉素钾0.10g与四环素0.10g作为灭菌溶液。
作为青霉素钾与四环素的替代品,也可以每1L培养基中加入50mg氯霉素。
\item 乳糖肉汤\\
\begin{tabular*}{3in}{l@{\extracolsep{\fill}}r}
肉浸出物&3.0g\\
明胶胨&5.0g\\
乳糖一水合物&5.0g\\
水&1000mL\\
\end{tabular*}
\\
混合所有成分,于121\textcelsius高压蒸气灭菌15$\sim$20分钟。灭菌后的pH6.7$\sim$7.1。
灭菌后迅速冷却。
\item 麦康凯琼脂培养基\\
\begin{tabular*}{3in}{l@{\extracolsep{\fill}}r}
明胶胨&17.0g\\
酪蛋白胨&1.5g\\
肉制蛋白胨&1.5g\\
乳糖一水合物&10.0g\\
去氧胆酸钠\endnote{}&1.5g\\
氯化钠&5.0g\\
琼脂&13.5g\\
中性红&0.03g\\
结晶紫\endnote{}&1.0mg\\
水&1000mL\\
\end{tabular*}
\\
混合所有成分,煮沸1分钟,混和后于121\textcelsius高压蒸气灭菌15$\sim$20分钟。
灭菌后的pH6.9$\sim$7.3。
\item EMB(曙红亚甲蓝)琼脂培养基\\
\begin{tabular*}{3in}{l@{\extracolsep{\fill}}r}
明胶胨&10.0g\\
磷酸氢二钾&2.0g\\
乳糖一水合物&10.0g\\
琼脂&15.0g\\
曙红Y&0.40g\\
亚甲蓝&0.065g\\
水&1000mL\\
\end{tabular*}
\\
混合所有成分,于121\textcelsius高压蒸气灭菌15$\sim$20分钟。灭菌后的pH6.9$\sim$7.3。
\item 液体亚硒酸盐胱氨酸培养基\\
\begin{tabular*}{3in}{l@{\extracolsep{\fill}}r}
明胶胨&5.0g\\
乳糖一水合物&4.0g\\
磷酸钠十二水合物&10.0g\\
亚硒酸钠&4.0g\\
$_L$-胱氨酸&0.010g\\
水&1000mL\\
\end{tabular*}
\\
混合所有成分,加温溶解。最终pH6.8$\sim$7.2。不要灭菌。
\item 液体四硫磺酸盐培养基\\
\begin{tabular*}{3in}{l@{\extracolsep{\fill}}r}
酪蛋白胨&2.5g\\
肉制蛋白胨&2.5g\\
去氧胆酸钠&1.0g\\
碳酸钙&10.0g\\
硫代硫酸钠五水合物&30.0g\\
水&1000mL\\
\end{tabular*}
\\
煮沸含固体的上述溶液。使用当天,加入20mL水中溶有碘化钾5g及碘6g的溶液。
再加入亮绿试液(1$\rightarrow$1000)10mL,混合。之后不要加热培养基。
\item Rappaport液体培养基\\
\begin{tabular*}{3in}{l@{\extracolsep{\fill}}r}
大豆胨&5.0g\\
氯化钠&8.0g\\
磷酸二氢钾&1.6g\\
孔雀绿&0.12g\\
氯化镁六水合物&40.0g\\
水&1000mL\\
\end{tabular*}
\\
将孔雀绿、氯化镁六水合物及其他成分分别溶于水中,于121\textcelsius高压蒸气灭菌15$\sim$20分钟。
灭菌后,混合使用。最终pH5.4$\sim$5.8。
\item 亮绿琼脂培养基\\
\begin{tabular*}{3in}{l@{\extracolsep{\fill}}r}
胨(动物组织制及酪蛋白制)&10.0g\\
酵母浸出物&3.0g\\
氯化钠&5.0g\\
乳糖一水合物&10.0g\\
蔗糖&10.0g\\
酚红&0.080g\\
亮绿&0.0125g\\
琼脂&20.0g\\
水&1000mL\\
\end{tabular*}
\\
混合所有成分,煮沸1分钟。使用前直接于121\textcelsius高压蒸气灭菌15$\sim$20分钟。
灭菌后的pH6.7$\sim$7.1。冷却至约50\textcelsius分注于培养皿中。
\item XLD(木糖$\cdot$赖氨酸$\cdot$去氧胆酸)琼脂培养基\\
\begin{tabular*}{3in}{l@{\extracolsep{\fill}}r}
$_D$-木糖&3.5g\\
$_L$-赖氨酸盐酸盐&5.0g\\
乳糖一水合物&7.5g\\
蔗糖&7.5g\\
氯化钠&5.0g\\
酵母浸出物&3.0g\\
酚红&0.080g\\
去氧胆酸钠&2.5g\\
硫代硫酸钠五水合物&6.8g\\
柠檬酸铁铵&0.80g\\
琼脂&13.5g\\
水&1000mL\\
\end{tabular*}
\\
混合所有成分,煮沸使溶解。煮沸后的pH7.2$\sim$7.6。不要高压蒸气灭菌。
避免过热。煮沸后,冷却至约50\textcelsius分注于培养皿中。
\item 亚硫酸铋琼脂培养基\\
\begin{tabular*}{3in}{l@{\extracolsep{\fill}}r}
肉浸出物&5.0g\\
酪蛋白胨&5.0g\\
肉制蛋白胨&5.0g\\
葡萄糖&5.0g\\
磷酸钠十二水合物&4.0g\\
硫酸亚铁七水合物&0.30g\\
亚硫酸铋指示剂&8.0g\\
亮绿&0.025g\\
琼脂&20.0g\\
水&1000mL\\
\end{tabular*}
\\
混合所有成分,煮沸使溶解。煮沸后的pH7.4$\sim$7.8。不要高压蒸气灭菌。
避免过热。煮沸后,冷却至约50\textcelsius分注于培养皿中。
\item TSI(三糖铁)琼脂培养基\\
\begin{tabular*}{3in}{l@{\extracolsep{\fill}}r}
酪蛋白胨&10.0g\\
肉制蛋白胨&10.0g\\
乳糖一水合物&10.0g\\
蔗糖&10.0g\\
葡萄糖&1.0g\\
硫酸亚铁铵六水合物&0.20g\\
氯化钠&5.0g\\
硫代硫酸钠五水合物&0.20g\\
酚红&0.025g\\
琼脂&13.0g\\
水&1000mL\\
\end{tabular*}
\\
混合所有成分,煮沸使溶解后,分注于小试管中,于121\textcelsius高压蒸气灭菌15$\sim$20分钟。
灭菌后pH7.1$\sim$7.5。用做斜面琼脂培养基。另外,可在上述成分中加入肉浸出物或
酵母浸出物3g,也可用柠檬酸铁铵代替硫酸亚铁铵六水合物使用。
\item 十六烷三甲基溴化铵琼脂培养基\\
\begin{tabular*}{3in}{l@{\extracolsep{\fill}}r}
明胶胨&20.0g\\
氯化镁六水合物&3.0g\\
硫酸钾&10.0g\\
十六烷三甲基溴化铵&0.30g\\
甘油&10mL\\
琼脂&13.6g\\
水&1000mL\\
\end{tabular*}
\\
混合所有成分溶于水中,加入甘油。加热并时时振摇,煮沸1分钟后,
于121\textcelsius高压蒸气灭菌15$\sim$20分钟。灭菌后的pH7.0$\sim$7.4。
\item NAC琼脂培养基\\
\begin{tabular*}{3in}{l@{\extracolsep{\fill}}r}
胨&20.0g\\
磷酸氢二钾&0.3g\\
硫酸镁七水合物&0.2g\\
十六烷三甲基溴化铵&0.2g\\
萘啶酸&0.015g\\
琼脂&15.0g\\
水&1000mL\\
\end{tabular*}
\\
最终pH7.2$\sim$7.6。不要灭菌。加热溶解。
\item 荧光素检出用假单胞杆菌属琼脂培养基\\
\begin{tabular*}{3in}{l@{\extracolsep{\fill}}r}
酪蛋白胨&10.0g\\
肉制蛋白胨&10.0g\\
磷酸氢二钾&1.5g\\
硫酸镁七水合物&1.5g\\
甘油&10mL\\
琼脂&15.0g\\
水&1000mL\\
\end{tabular*}
\\
混合所有成分溶于水中,加入甘油。加热并时时振摇,煮沸1分钟后,
于121\textcelsius高压蒸气灭菌15$\sim$20分钟。灭菌后的pH7.0$\sim$7.4。
\item 绿脓菌素检出用假单胞杆菌属琼脂培养基\\
\begin{tabular*}{3in}{l@{\extracolsep{\fill}}r}
明胶胨&20.0g\\
氯化镁六水合物&3.0g\\
硫酸钾&10.0g\\
甘油&10mL\\
琼脂&15.0g\\
水&1000mL\\
\end{tabular*}
\\
混合所有成分溶于水中,加入甘油。加热并时时振摇,煮沸1分钟后,
于121\textcelsius高压蒸气灭菌15$\sim$20分钟。灭菌后的pH7.0$\sim$7.4。
\item Vogel-Johnson琼脂培养基\\
\begin{tabular*}{3in}{l@{\extracolsep{\fill}}r}
酪蛋白胨&10.0g\\
酵母浸出物&5.0g\\
$_D$-甘露醇&10.0g\\
磷酸氢二钾&5.0g\\
氯化锂&5.0g\\
甘氨酸&10.0g\\
酚红&0.025g\\
琼脂&16.0g\\
水&1000mL\\
\end{tabular*}
\\
混合所有成分,煮沸1分钟使溶解后。于121\textcelsius高压蒸气灭菌15$\sim$20分钟。
冷却至45$\sim$50\textcelsius。灭菌后pH7.0$\sim$7.4。在其中加入已灭菌的亚碲酸钾溶液
(1$\rightarrow$100)20mL,混合。
\item Baid-Parker琼脂培养基\\
\begin{tabular*}{3in}{l@{\extracolsep{\fill}}r}
酪蛋白胨&10.0g\\
肉浸出物&5.0g\\
酵母浸出物&1.0g\\
氯化锂&5.0g\\
甘氨酸&12.0g\\
丙酮酸钠\endnote{}&10.0g\\
琼脂&20.0g\\
水&950mL\\
\end{tabular*}
\\
混合所有成分,时时振摇加热,煮沸1分钟。于121\textcelsius高压蒸气灭菌15$\sim$20分钟后,
冷却至45$\sim$50\textcelsius。灭菌后pH6.6$\sim$7.0。在其中加入已灭菌的亚碲酸钾溶液
(1$\rightarrow$100)10mL与50mL卵黄乳浊液,轻轻混合后,分注于培养皿中。
卵黄乳浊液为按卵黄约30\%,生理盐水约70\%的比例混合制备而成。
\item 甘露醇氯化钠琼脂培养基\\
\begin{tabular*}{3in}{l@{\extracolsep{\fill}}r}
酪蛋白胨&5.0g\\
肉制蛋白胨&5.0g\\
肉浸出物&1.0g\\
$_D$-甘露醇&10.0g\\
氯化钠&75.0g\\
酚红&0.025g\\
琼脂&15.0g\\
水&1000mL\\
\end{tabular*}
\\
混合所有成分,时时振摇加热,煮沸1分钟后,于121\textcelsius高压蒸气灭菌15$\sim$20分钟。
灭菌后pH7.2$\sim$7.6。
\end{enumerate}

\\
\begin{center}\textbf{(注及解说略去。)}\end{center}

%todo:在这里输出endnotes:
%\theendnotes

%Trick: 用\newpage命令告诉latex在此时放上heading,否则,latex
%会先关闭CJK环境,再放heading,如果heading中有CJK字符即无法显示了。
\newpage

\end{SC}
\end{CJK}

\end{document}
